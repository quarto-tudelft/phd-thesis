% Options for packages loaded elsewhere
% Options for packages loaded elsewhere
\PassOptionsToPackage{unicode}{hyperref}
\PassOptionsToPackage{hyphens}{url}
\PassOptionsToPackage{dvipsnames,svgnames,x11names}{xcolor}
%
\documentclass{tudelft}
\usepackage{xcolor}
\usepackage{amsmath,amssymb}
\setcounter{secnumdepth}{5}
\usepackage{iftex}
\ifPDFTeX
  \usepackage[T1]{fontenc}
  \usepackage[utf8]{inputenc}
  \usepackage{textcomp} % provide euro and other symbols
\else % if luatex or xetex
  \usepackage{unicode-math} % this also loads fontspec
  \defaultfontfeatures{Scale=MatchLowercase}
  \defaultfontfeatures[\rmfamily]{Ligatures=TeX,Scale=1}
\fi
\usepackage{lmodern}
\ifPDFTeX\else
  % xetex/luatex font selection
\fi
% Use upquote if available, for straight quotes in verbatim environments
\IfFileExists{upquote.sty}{\usepackage{upquote}}{}
\IfFileExists{microtype.sty}{% use microtype if available
  \usepackage[]{microtype}
  \UseMicrotypeSet[protrusion]{basicmath} % disable protrusion for tt fonts
}{}
\makeatletter
\@ifundefined{KOMAClassName}{% if non-KOMA class
  \IfFileExists{parskip.sty}{%
    \usepackage{parskip}
  }{% else
    \setlength{\parindent}{0pt}
    \setlength{\parskip}{6pt plus 2pt minus 1pt}}
}{% if KOMA class
  \KOMAoptions{parskip=half}}
\makeatother
% Make \paragraph and \subparagraph free-standing
\makeatletter
\ifx\paragraph\undefined\else
  \let\oldparagraph\paragraph
  \renewcommand{\paragraph}{
    \@ifstar
      \xxxParagraphStar
      \xxxParagraphNoStar
  }
  \newcommand{\xxxParagraphStar}[1]{\oldparagraph*{#1}\mbox{}}
  \newcommand{\xxxParagraphNoStar}[1]{\oldparagraph{#1}\mbox{}}
\fi
\ifx\subparagraph\undefined\else
  \let\oldsubparagraph\subparagraph
  \renewcommand{\subparagraph}{
    \@ifstar
      \xxxSubParagraphStar
      \xxxSubParagraphNoStar
  }
  \newcommand{\xxxSubParagraphStar}[1]{\oldsubparagraph*{#1}\mbox{}}
  \newcommand{\xxxSubParagraphNoStar}[1]{\oldsubparagraph{#1}\mbox{}}
\fi
\makeatother

\usepackage{color}
\usepackage{fancyvrb}
\newcommand{\VerbBar}{|}
\newcommand{\VERB}{\Verb[commandchars=\\\{\}]}
\DefineVerbatimEnvironment{Highlighting}{Verbatim}{commandchars=\\\{\}}
% Add ',fontsize=\small' for more characters per line
\usepackage{framed}
\definecolor{shadecolor}{RGB}{241,243,245}
\newenvironment{Shaded}{\begin{snugshade}}{\end{snugshade}}
\newcommand{\AlertTok}[1]{\textcolor[rgb]{0.68,0.00,0.00}{#1}}
\newcommand{\AnnotationTok}[1]{\textcolor[rgb]{0.37,0.37,0.37}{#1}}
\newcommand{\AttributeTok}[1]{\textcolor[rgb]{0.40,0.45,0.13}{#1}}
\newcommand{\BaseNTok}[1]{\textcolor[rgb]{0.68,0.00,0.00}{#1}}
\newcommand{\BuiltInTok}[1]{\textcolor[rgb]{0.00,0.23,0.31}{#1}}
\newcommand{\CharTok}[1]{\textcolor[rgb]{0.13,0.47,0.30}{#1}}
\newcommand{\CommentTok}[1]{\textcolor[rgb]{0.37,0.37,0.37}{#1}}
\newcommand{\CommentVarTok}[1]{\textcolor[rgb]{0.37,0.37,0.37}{\textit{#1}}}
\newcommand{\ConstantTok}[1]{\textcolor[rgb]{0.56,0.35,0.01}{#1}}
\newcommand{\ControlFlowTok}[1]{\textcolor[rgb]{0.00,0.23,0.31}{\textbf{#1}}}
\newcommand{\DataTypeTok}[1]{\textcolor[rgb]{0.68,0.00,0.00}{#1}}
\newcommand{\DecValTok}[1]{\textcolor[rgb]{0.68,0.00,0.00}{#1}}
\newcommand{\DocumentationTok}[1]{\textcolor[rgb]{0.37,0.37,0.37}{\textit{#1}}}
\newcommand{\ErrorTok}[1]{\textcolor[rgb]{0.68,0.00,0.00}{#1}}
\newcommand{\ExtensionTok}[1]{\textcolor[rgb]{0.00,0.23,0.31}{#1}}
\newcommand{\FloatTok}[1]{\textcolor[rgb]{0.68,0.00,0.00}{#1}}
\newcommand{\FunctionTok}[1]{\textcolor[rgb]{0.28,0.35,0.67}{#1}}
\newcommand{\ImportTok}[1]{\textcolor[rgb]{0.00,0.46,0.62}{#1}}
\newcommand{\InformationTok}[1]{\textcolor[rgb]{0.37,0.37,0.37}{#1}}
\newcommand{\KeywordTok}[1]{\textcolor[rgb]{0.00,0.23,0.31}{\textbf{#1}}}
\newcommand{\NormalTok}[1]{\textcolor[rgb]{0.00,0.23,0.31}{#1}}
\newcommand{\OperatorTok}[1]{\textcolor[rgb]{0.37,0.37,0.37}{#1}}
\newcommand{\OtherTok}[1]{\textcolor[rgb]{0.00,0.23,0.31}{#1}}
\newcommand{\PreprocessorTok}[1]{\textcolor[rgb]{0.68,0.00,0.00}{#1}}
\newcommand{\RegionMarkerTok}[1]{\textcolor[rgb]{0.00,0.23,0.31}{#1}}
\newcommand{\SpecialCharTok}[1]{\textcolor[rgb]{0.37,0.37,0.37}{#1}}
\newcommand{\SpecialStringTok}[1]{\textcolor[rgb]{0.13,0.47,0.30}{#1}}
\newcommand{\StringTok}[1]{\textcolor[rgb]{0.13,0.47,0.30}{#1}}
\newcommand{\VariableTok}[1]{\textcolor[rgb]{0.07,0.07,0.07}{#1}}
\newcommand{\VerbatimStringTok}[1]{\textcolor[rgb]{0.13,0.47,0.30}{#1}}
\newcommand{\WarningTok}[1]{\textcolor[rgb]{0.37,0.37,0.37}{\textit{#1}}}

\usepackage{longtable,booktabs,array}
\usepackage{calc} % for calculating minipage widths
% Correct order of tables after \paragraph or \subparagraph
\usepackage{etoolbox}
\makeatletter
\patchcmd\longtable{\par}{\if@noskipsec\mbox{}\fi\par}{}{}
\makeatother
% Allow footnotes in longtable head/foot
\IfFileExists{footnotehyper.sty}{\usepackage{footnotehyper}}{\usepackage{footnote}}
\makesavenoteenv{longtable}
\usepackage{graphicx}
\makeatletter
\newsavebox\pandoc@box
\newcommand*\pandocbounded[1]{% scales image to fit in text height/width
  \sbox\pandoc@box{#1}%
  \Gscale@div\@tempa{\textheight}{\dimexpr\ht\pandoc@box+\dp\pandoc@box\relax}%
  \Gscale@div\@tempb{\linewidth}{\wd\pandoc@box}%
  \ifdim\@tempb\p@<\@tempa\p@\let\@tempa\@tempb\fi% select the smaller of both
  \ifdim\@tempa\p@<\p@\scalebox{\@tempa}{\usebox\pandoc@box}%
  \else\usebox{\pandoc@box}%
  \fi%
}
% Set default figure placement to htbp
\def\fps@figure{htbp}
\makeatother





\setlength{\emergencystretch}{3em} % prevent overfull lines

\providecommand{\tightlist}{%
  \setlength{\itemsep}{0pt}\setlength{\parskip}{0pt}}



 


\makeatletter
\@ifpackageloaded{tcolorbox}{}{\usepackage[skins,breakable]{tcolorbox}}
\@ifpackageloaded{fontawesome5}{}{\usepackage{fontawesome5}}
\definecolor{quarto-callout-color}{HTML}{909090}
\definecolor{quarto-callout-note-color}{HTML}{0758E5}
\definecolor{quarto-callout-important-color}{HTML}{CC1914}
\definecolor{quarto-callout-warning-color}{HTML}{EB9113}
\definecolor{quarto-callout-tip-color}{HTML}{00A047}
\definecolor{quarto-callout-caution-color}{HTML}{FC5300}
\definecolor{quarto-callout-color-frame}{HTML}{acacac}
\definecolor{quarto-callout-note-color-frame}{HTML}{4582ec}
\definecolor{quarto-callout-important-color-frame}{HTML}{d9534f}
\definecolor{quarto-callout-warning-color-frame}{HTML}{f0ad4e}
\definecolor{quarto-callout-tip-color-frame}{HTML}{02b875}
\definecolor{quarto-callout-caution-color-frame}{HTML}{fd7e14}
\makeatother
\makeatletter
\@ifpackageloaded{bookmark}{}{\usepackage{bookmark}}
\makeatother
\makeatletter
\@ifpackageloaded{caption}{}{\usepackage{caption}}
\AtBeginDocument{%
\ifdefined\contentsname
  \renewcommand*\contentsname{Table of contents}
\else
  \newcommand\contentsname{Table of contents}
\fi
\ifdefined\listfigurename
  \renewcommand*\listfigurename{List of Figures}
\else
  \newcommand\listfigurename{List of Figures}
\fi
\ifdefined\listtablename
  \renewcommand*\listtablename{List of Tables}
\else
  \newcommand\listtablename{List of Tables}
\fi
\ifdefined\figurename
  \renewcommand*\figurename{Figure}
\else
  \newcommand\figurename{Figure}
\fi
\ifdefined\tablename
  \renewcommand*\tablename{Table}
\else
  \newcommand\tablename{Table}
\fi
}
\@ifpackageloaded{float}{}{\usepackage{float}}
\floatstyle{ruled}
\@ifundefined{c@chapter}{\newfloat{codelisting}{h}{lop}}{\newfloat{codelisting}{h}{lop}[chapter]}
\floatname{codelisting}{Listing}
\newcommand*\listoflistings{\listof{codelisting}{List of Listings}}
\makeatother
\makeatletter
\makeatother
\makeatletter
\@ifpackageloaded{caption}{}{\usepackage{caption}}
\@ifpackageloaded{subcaption}{}{\usepackage{subcaption}}
\makeatother
\makeatletter
\@ifpackageloaded{algorithm}{}{\usepackage{algorithm}}
\makeatother
\makeatletter
\@ifpackageloaded{algpseudocode}{}{\usepackage{algpseudocode}}
\makeatother
\makeatletter
\@ifpackageloaded{caption}{}{\usepackage{caption}}
\makeatother
\makeatletter
\@ifpackageloaded{tcolorbox}{}{\usepackage[many]{tcolorbox}}
\makeatother
%%%% ---foldboxy preamble ----- %%%%%

\definecolor{fbx-default-color1}{HTML}{c7c7d0}
\definecolor{fbx-default-color2}{HTML}{a3a3aa}

\definecolor{fbox-color1}{HTML}{c7c7d0}
\definecolor{fbox-color2}{HTML}{a3a3aa}

% arguments: #1 typelabelnummer: #2 titel: #3
\newenvironment{fbx}[3]{\begin{tcolorbox}[enhanced, breakable,%
attach boxed title to top*={xshift=1.4pt},
boxed title style={boxrule=0.0mm, fuzzy shadow={1pt}{-1pt}{0mm}{0.1mm}{gray}, arc=.3em, rounded corners=east, sharp corners=west}, colframe=#1-color2, colbacktitle=#1-color1, colback = white, coltitle=black,  titlerule=0mm, toprule=0pt, bottomrule=.7pt, leftrule=.3em, rightrule=0pt, outer arc=.3em,  arc=0pt,	 sharp corners = east, left=.5em, bottomtitle=1mm, toptitle=1mm,title=\textbf{#2}\hspace{0.5em}{#3}]}
{\end{tcolorbox}}

% boxed environment with right border
\newenvironment{fbxSimple}[3]{\begin{tcolorbox}[enhanced, breakable,%
attach boxed title to top*={xshift=1.4pt},
boxed title style={boxrule=0.0mm, fuzzy shadow={1pt}{-1pt}{0mm}{0.1mm}{gray}, arc=.3em, rounded corners=east, sharp corners=west}, colframe=#1-color2, colbacktitle=#1-color1, colback = white, coltitle=black,  titlerule=0mm, toprule=0pt, bottomrule=.7pt, leftrule=.3em, rightrule=.7pt, outer arc=.3em,  	left=.5em, right=.5em, bottomtitle=1mm, toptitle=1mm,title=\textbf{#2}\hspace{0.5em}{#3}]}
{\end{tcolorbox}}

%%%% --- end foldboxy preamble ----- %%%%%
%%==== colors from yaml ===%
\definecolor{RQ-color1}{HTML}{dff4fa}
\definecolor{RQ-color2}{HTML}{00a6d6}
\definecolor{TRQ-color1}{HTML}{80d2eb}
\definecolor{TRQ-color2}{HTML}{00a6d6}
\definecolor{Rec-color1}{HTML}{fff6e3}
\definecolor{Rec-color2}{HTML}{ffb81c}
%=============%
\usepackage{bookmark}
\IfFileExists{xurl.sty}{\usepackage{xurl}}{} % add URL line breaks if available
\urlstyle{same}
\hypersetup{
  pdftitle={My Thesis},
  pdfkeywords={Quarto, Extension, Template},
  colorlinks=true,
  linkcolor={blue},
  filecolor={Maroon},
  citecolor={Blue},
  urlcolor={Blue},
  pdfcreator={LaTeX via pandoc}}


\title{My Thesis}
\author{}
\date{}
\begin{document}
% Adapted with the help of Claude Sonnet 3.7
% !TeX spellcheck = en_GB

% Apply frontmatter styling (Roman numerals for the page numbers of the title pages and toc)
\frontmatter

\begin{titlepage}

    \begin{center}

        %% Extra whitespace at the top.
        \vspace*{2\bigskipamount}

        %% Print the title.
        {\makeatletter
            \titlestyle\bfseries\LARGE\@title
            \makeatother}

        %% Print the optional subtitle.
        {\makeatletter
            \ifx\@subtitle\undefined\else
                \bigskip
                \titlefont\titleshape\Large\@subtitle
            \fi
            \makeatother}

    \end{center}

    \cleardoublepage
    \thispagestyle{empty}

    \begin{center}

        %% The following lines repeat the previous page exactly.

        \vspace*{2\bigskipamount}

        %% Print the title.
        {\makeatletter
            \titlestyle\bfseries\LARGE\@title
            \makeatother}

        %% Print the optional subtitle.
        {\makeatletter
            \ifx\@subtitle\undefined\else
                \bigskip
                \titlefont\titleshape\Large\@subtitle
            \fi
            \makeatother}

        %% Uncomment the following lines to insert a vertically centered picture into
        %% the title page.
        %\vfill
        %\includegraphics{title}
        \vfill

        %% Apart from the names and dates, the following text is dictated by the
        %% promotieregelement.

        {\Large\titlefont\bfseries Dissertation}

        \bigskip
        \bigskip

        for the purpose of obtaining the degree of doctor

        at Delft University of Technology

        by the authority of the Rector Magnificus, Prof.~dr.~ir.~T.H.J.J.~van der Hagen,

        Chair of the Board for Doctorates

        to be defended publicly on

        Tuesday 20, June 2025 at 10:00 o'clock

        \bigskip
        \bigskip

        by

        \bigskip
        \bigskip

        %% Print the full name of the author.
        \author{}
        
        \bigskip
        \bigskip

        MSc, University of Research, Country

        born in Amsterdam, The Netherlands

        %% Extra whitespace at the bottom.
        \vspace*{2\bigskipamount}

    \end{center}

    \clearpage
    \thispagestyle{empty}

    %% The following line is dictated by the promotieregelement.
    \noindent This dissertation has been approved by the promotors.

    \bigskip
    \noindent Composition of the doctoral committee:
    %% List the committee members, starting with the Rector Magnificus and the
    %% promotor(s) and ending with the reserve members.
    \begin{tabbing}
        \hspace{\tabcolsep}\=\hspace{0.33\textwidth}\=\hspace{0.66\textwidth}                   \\[-3\medskipamount]
        \> Rector Magnificus,          \> chairperson\\
        \> Prof.~dr. A. Kleiner,    \> Delft University of
Technology, \textit{promotor}    \\
        \> Dr.~A.A. Aaronson,    \> Delft University of
Technology, \textit{copromotor}    \\
        \>\textit{Independent members:}                                                        \\[\smallskipamount]
        \>Prof.~dr. A. Jansen       \> Delft University of
Technology                         \\
        \>Prof.~dr. ir. A.B.C.D. van de Lange-Achternaam\\
        \>       \> Delft University of
Technology                         \\
        \>Prof.~dr. N. Nescio       \> Politecnico di Milano,
Italy                         \\
        \>Prof.~dr. ir. J. Doe       \> Delft University of
Technology, reserve member                         \\
        \>\textit{Other members:}                                                               \\[\smallskipamount]
        \>Prof.~dr. ir. J. de Wit, \> Delft University of
Technology                         \\
        \>Dr.~ir. Q. de Zwart, \> Delft University of
Technology                         \\
    \end{tabbing}

    %% Include the following disclaimer for committee members who have contributed
    %% to this dissertation. Its formulation is again dictated by the
    %% promotieregelement.

    \medskip
    \noindent This is an optional message to acknowledge partner
institutes.

    %% Here you can include the logos of any institute that contributed financially
    %% to this dissertation.
    \vfill
        \begin{center}
                                \includegraphics[height=0.75in]{www/tudelft.png}\hspace{1.5cm}
                        \includegraphics[height=0.75in]{www/delft-final.png}\hspace{1cm}
                        \end{center}
    \vfill

    %% Here you can include the logos of any institute that contributed financially
    %% to this dissertation.
    \vfill
    \begin{center}
    \end{center}
    \vfill

    \noindent
    \begin{tabular}{@{}p{0.2\textwidth}@{}p{0.8\textwidth}@{}}
        \textit{Keywords:}    & Quarto, Extension, Template \\[\medskipamount]
        \textit{Printed by:}   & Johannes Gutenberg \\[\medskipamount]
        \textit{Cover by:} & Beautiful cover art that captures the entire content of this thesis in a single illustration.
    \end{tabular}

    \vspace{4\bigskipamount}



    \medskip
    \noindent The author set this thesis using Quarto: \url{https://github.com/quarto-tudelft}.

    \medskip
    \noindent ISBN 000-00-0000-000-0

    \medskip
    \noindent An electronic copy of this dissertation is available at\\
    \url{https://repository.tudelft.nl/}.

\end{titlepage}

%% The (optional) dedication can be used to thank someone or display a
%% significant quotation.
\dedication{
      \epigraph{They not like us.}{Kendrick Lamar}
  }

%% Chapter-level dedication

{
  \cleardoublepage%
  \phantomsection%
}
\floatname{algorithm}{Algorithm}

\numberwithin{algorithm}{chapter}

\renewcommand*\contentsname{Table of contents}
{
\hypersetup{linkcolor=}
\setcounter{tocdepth}{2}
\tableofcontents
}

\bookmarksetup{startatroot}

\chapter*{Preface}\label{preface}
\addcontentsline{toc}{chapter}{Preface}

\markboth{Preface}{Preface}

\begin{tcolorbox}[enhanced jigsaw, titlerule=0mm, leftrule=.75mm, colbacktitle=quarto-callout-note-color!10!white, colback=white, opacityback=0, opacitybacktitle=0.6, toprule=.15mm, arc=.35mm, breakable, toptitle=1mm, coltitle=black, title=\textcolor{quarto-callout-note-color}{\faInfo}\hspace{0.5em}{Note}, rightrule=.15mm, bottomrule=.15mm, bottomtitle=1mm, left=2mm, colframe=quarto-callout-note-color-frame]

Preface goes here. This chapter is optional.

\end{tcolorbox}

This is the TU Delft dissertation template for Quarto use the book
project type. To learn more about Quarto books visit
\url{https://quarto.org/docs/books}.

The template uses the \texttt{quarto-tudelft} extension for formatting.
The extension itself is adapted from a LaTeX template that can be found
at GitLab:
\href{https://gitlab.com/novanext/tudelft-dissertation}{novanext/tudelft-dissertation}.

\bookmarksetup{startatroot}

\chapter*{Summary}\label{summary}
\addcontentsline{toc}{chapter}{Summary}

\markboth{Summary}{Summary}

In summary, this book has no content whatsoever.

\bookmarksetup{startatroot}

\chapter*{Samenvatting}\label{samenvatting}
\addcontentsline{toc}{chapter}{Samenvatting}

\markboth{Samenvatting}{Samenvatting}

\mainmatter

\bookmarksetup{startatroot}

\chapter{Introduction}\label{introduction}

This is a book created from markdown and executable code.

\bookmarksetup{startatroot}

\chapter{Guide}\label{guide}

Chapter abstract.

\hfill\break

\section{Format Options}\label{format-options}

This extension provides several variables to customize your thesis title
page:

\begin{longtable}[]{@{}
  >{\raggedright\arraybackslash}p{(\linewidth - 4\tabcolsep) * \real{0.3125}}
  >{\raggedright\arraybackslash}p{(\linewidth - 4\tabcolsep) * \real{0.4062}}
  >{\raggedright\arraybackslash}p{(\linewidth - 4\tabcolsep) * \real{0.2812}}@{}}
\toprule\noalign{}
\begin{minipage}[b]{\linewidth}\raggedright
Variable
\end{minipage} & \begin{minipage}[b]{\linewidth}\raggedright
Description
\end{minipage} & \begin{minipage}[b]{\linewidth}\raggedright
Example
\end{minipage} \\
\midrule\noalign{}
\endhead
\bottomrule\noalign{}
\endlastfoot
\texttt{title} & Title of your dissertation & ``My Doctoral
Dissertation'' \\
\texttt{subtitle} & Optional subtitle & ``A Study of Something
Important'' \\
\texttt{author} & Full name of the author & ``Jane Doe'' \\
\texttt{author-degree} & Highest academic title, university, country &
``MSc, University of Research, Country'' \\
\texttt{author-birthplace} & Town/city and country of birth &
``Amsterdam, The Netherlands'' \\
\texttt{defense-date} & Date and time of the defense & ``Tuesday 20,
June 2025 at 10:00 o'clock'' \\
\texttt{rector} & Name of the Rector Magnificus & ``prof. dr. ir. A. B.
Example'' \\
\texttt{committee} & List of committee members & See example below \\
\texttt{independent-members} & List of independent members & See example
below \\
\texttt{other-members} & List of other members & See example below \\
\texttt{contributors} & List of committee members who contributed to the
dissertation & See example below \\
\texttt{keywords} & Keywords describing the research & ``keyword1,
keyword2, keyword3'' \\
\texttt{printed-by} & Printing company information & ``University
Press'' \\
\texttt{cover-by} & Cover design credit & ``Jane Smith'' \\
\texttt{copyright-year} & Year of copyright & ``2025'' \\
\texttt{phd-series} & PhD series information if applicable & ``Casimir
PhD Series, Delft-Leiden 2025-01'' \\
\texttt{isbn} & ISBN number & ``978-90-8593-000-0'' \\
\texttt{repository-url} & URL to the electronic version &
``https://repository.tudelft.nl/'' \\
\end{longtable}

\subsection{TU Delft Logo Inclusion}\label{tu-delft-logo-inclusion}

The extension automatically includes the TU Delft logo by default. You
can control this behavior:

\begin{Shaded}
\begin{Highlighting}[]
\CommentTok{\# Include the TU Delft logo (default: true)}
\FunctionTok{institute{-}logo{-}tudelft}\KeywordTok{:}\AttributeTok{ }\CharTok{true}\AttributeTok{  }

\CommentTok{\# Add additional logos}
\FunctionTok{institute{-}logos}\KeywordTok{:}
\AttributeTok{  }\KeywordTok{{-}}\AttributeTok{ }\FunctionTok{path}\KeywordTok{:}\AttributeTok{ }\StringTok{"images/partner{-}institute{-}logo.png"}
\AttributeTok{  }\KeywordTok{{-}}\AttributeTok{ }\FunctionTok{path}\KeywordTok{:}\AttributeTok{ }\StringTok{"images/another{-}institute{-}logo.png"}
\end{Highlighting}
\end{Shaded}

To disable the automatic inclusion of the TU Delft logo:

\begin{Shaded}
\begin{Highlighting}[]
\FunctionTok{institute{-}logo{-}tudelft}\KeywordTok{:}\AttributeTok{ }\CharTok{false}
\end{Highlighting}
\end{Shaded}

When providing your own logos, the TU Delft logo (if enabled) will
always appear first, followed by your specified logos.

\subsection{Multiple Institute Logos}\label{multiple-institute-logos}

To include multiple logos of institutions that contributed to your
research:

\begin{Shaded}
\begin{Highlighting}[]
\FunctionTok{institute{-}logos}\KeywordTok{:}
\AttributeTok{  }\KeywordTok{{-}}\AttributeTok{ }\FunctionTok{path}\KeywordTok{:}\AttributeTok{ }\StringTok{"images/tudelft{-}logo.png"}
\AttributeTok{    }\FunctionTok{spacing}\KeywordTok{:}\AttributeTok{ }\StringTok{"1.5cm"}\CommentTok{  \# Optional spacing between logos}
\AttributeTok{  }\KeywordTok{{-}}\AttributeTok{ }\FunctionTok{path}\KeywordTok{:}\AttributeTok{ }\StringTok{"images/partner{-}institute{-}logo.png"}
\end{Highlighting}
\end{Shaded}

Alternatively, for a single logo, you can use the simpler syntax:

\begin{Shaded}
\begin{Highlighting}[]
\FunctionTok{institute{-}logo}\KeywordTok{:}\AttributeTok{ }\StringTok{"images/tudelft{-}logo.png"}
\end{Highlighting}
\end{Shaded}

\subsection{Example YAML
Configuration}\label{example-yaml-configuration}

\begin{Shaded}
\begin{Highlighting}[]
\PreprocessorTok{{-}{-}{-}}
\FunctionTok{title}\KeywordTok{:}\AttributeTok{ }\StringTok{"My Doctoral Dissertation"}
\FunctionTok{subtitle}\KeywordTok{:}\AttributeTok{ }\StringTok{"A Study of Something Important"}
\FunctionTok{author}\KeywordTok{:}\AttributeTok{ }\StringTok{"Jane Doe"}
\FunctionTok{author{-}degree}\KeywordTok{:}\AttributeTok{ }\StringTok{"MSc, University of Research, Country"}
\FunctionTok{author{-}birthplace}\KeywordTok{:}\AttributeTok{ }\StringTok{"Amsterdam, The Netherlands"}
\FunctionTok{defense{-}date}\KeywordTok{:}\AttributeTok{ }\StringTok{"Tuesday 20, June 2025 at 10:00 o\textquotesingle{}clock"}
\FunctionTok{rector}\KeywordTok{:}\AttributeTok{ }\StringTok{"prof. dr. ir. A. B. Example"}

\CommentTok{\# Committee members}
\FunctionTok{committee}\KeywordTok{:}
\AttributeTok{  }\KeywordTok{{-}}\AttributeTok{ }\FunctionTok{title}\KeywordTok{:}\AttributeTok{ }\StringTok{"Prof. dr. A. Kleiner"}
\AttributeTok{    }\FunctionTok{affiliation}\KeywordTok{:}\AttributeTok{ }\StringTok{"Delft University of Technology"}
\AttributeTok{    }\FunctionTok{role}\KeywordTok{:}\AttributeTok{ }\StringTok{"promotor"}
\AttributeTok{  }\KeywordTok{{-}}\AttributeTok{ }\FunctionTok{title}\KeywordTok{:}\AttributeTok{ }\StringTok{"Dr. A.A. Aaronson"}
\AttributeTok{    }\FunctionTok{affiliation}\KeywordTok{:}\AttributeTok{ }\StringTok{"Delft University of Technology"}
\AttributeTok{    }\FunctionTok{role}\KeywordTok{:}\AttributeTok{ }\StringTok{"copromotor"}

\FunctionTok{independent{-}members}\KeywordTok{:}
\AttributeTok{  }\KeywordTok{{-}}\AttributeTok{ }\FunctionTok{title}\KeywordTok{:}\AttributeTok{ }\StringTok{"Prof. dr. A. Jansen"}
\AttributeTok{    }\FunctionTok{affiliation}\KeywordTok{:}\AttributeTok{ }\StringTok{"Delft University of Technology"}
\AttributeTok{  }\KeywordTok{{-}}\AttributeTok{ }\FunctionTok{title}\KeywordTok{:}\AttributeTok{ }\StringTok{"Prof. dr. ir. A.B.C.D. van de Lange{-}Achternaam"}
\AttributeTok{    }\FunctionTok{affiliation}\KeywordTok{:}\AttributeTok{ }\StringTok{"Delft University of Technology"}
\AttributeTok{    }\FunctionTok{name{-}too{-}long}\KeywordTok{:}\AttributeTok{ }\CharTok{true}\CommentTok{  \# Use this for names that need to be on their own line}
\AttributeTok{  }\KeywordTok{{-}}\AttributeTok{ }\FunctionTok{title}\KeywordTok{:}\AttributeTok{ }\StringTok{"Prof. dr. N. Nescio"}
\AttributeTok{    }\FunctionTok{affiliation}\KeywordTok{:}\AttributeTok{ }\StringTok{"Politecnico di Milano, Italy"}
\AttributeTok{  }\KeywordTok{{-}}\AttributeTok{ }\FunctionTok{title}\KeywordTok{:}\AttributeTok{ }\StringTok{"Prof. dr. ir. J. Doe"}
\AttributeTok{    }\FunctionTok{affiliation}\KeywordTok{:}\AttributeTok{ }\StringTok{"Delft University of Technology"}
\AttributeTok{    }\FunctionTok{role}\KeywordTok{:}\AttributeTok{ }\StringTok{"reserve member"}

\FunctionTok{other{-}members}\KeywordTok{:}
\AttributeTok{  }\KeywordTok{{-}}\AttributeTok{ }\FunctionTok{title}\KeywordTok{:}\AttributeTok{ }\StringTok{"Prof. dr. ir. J. de Wit"}
\AttributeTok{    }\FunctionTok{affiliation}\KeywordTok{:}\AttributeTok{ }\StringTok{"Delft University of Technology"}
\AttributeTok{  }\KeywordTok{{-}}\AttributeTok{ }\FunctionTok{title}\KeywordTok{:}\AttributeTok{ }\StringTok{"Dr. ir. Q. de Zwart"}
\AttributeTok{    }\FunctionTok{affiliation}\KeywordTok{:}\AttributeTok{ }\StringTok{"Delft University of Technology"}

\CommentTok{\# Contributors to dissertation}
\FunctionTok{contributors}\KeywordTok{:}
\AttributeTok{  }\KeywordTok{{-}}\AttributeTok{ }\FunctionTok{title}\KeywordTok{:}\AttributeTok{ }\StringTok{"Prof. dr. ir. J. de Wit"}
\AttributeTok{    }\FunctionTok{affiliation}\KeywordTok{:}\AttributeTok{ }\StringTok{"Delft University of Technology"}

\CommentTok{\# Institute logos (choose one approach)}
\FunctionTok{institute{-}logos}\KeywordTok{:}
\AttributeTok{  }\KeywordTok{{-}}\AttributeTok{ }\FunctionTok{path}\KeywordTok{:}\AttributeTok{ }\StringTok{"images/tudelft{-}logo.png"}
\AttributeTok{    }\FunctionTok{spacing}\KeywordTok{:}\AttributeTok{ }\StringTok{"1.5cm"}
\AttributeTok{  }\KeywordTok{{-}}\AttributeTok{ }\FunctionTok{path}\KeywordTok{:}\AttributeTok{ }\StringTok{"images/partner{-}institute{-}logo.png"}

\CommentTok{\# Publication details}
\FunctionTok{keywords}\KeywordTok{:}\AttributeTok{ }\StringTok{"keyword1, keyword2, keyword3"}
\FunctionTok{printed{-}by}\KeywordTok{:}\AttributeTok{ }\StringTok{"University Press"}
\FunctionTok{cover{-}by}\KeywordTok{:}\AttributeTok{ }\StringTok{"Jane Smith, featuring an abstract representation of the research topic"}
\FunctionTok{isbn}\KeywordTok{:}\AttributeTok{ }\StringTok{"978{-}90{-}8593{-}000{-}0"}

\FunctionTok{format}\KeywordTok{:}
\AttributeTok{  }\FunctionTok{tudelft{-}thesis{-}pdf}\KeywordTok{:}\AttributeTok{ default}
\PreprocessorTok{{-}{-}{-}}
\end{Highlighting}
\end{Shaded}

\bookmarksetup{startatroot}

\chapter{Cool Quarto Things}\label{cool-quarto-things}

\section{Custom Callouts}\label{custom-callouts}

The template adds custom numbered blocks using the embedded
\href{https://github.com/ute/custom-numbered-blocks}{ute/custom-numbered-blocks}
extension, which supports HTML and PDF with non-standard
cross-references (i.e.~\texttt{\textbackslash{}ref\{\}} syntax). We
provide two custom numbered blocks:

We provide two custom numbered blocks:

\begin{enumerate}
\def\labelenumi{\arabic{enumi}.}
\tightlist
\item
  \textbf{TRQ} for Thesis Research Questions
\item
  \textbf{RQ} for Research Questions
\item
  \textbf{Rec} for Recommendations
\end{enumerate}

TRQ \hyperref[trq:what]{3.1} is an example of a Thesis Research
Question; RQ \hyperref[rq:what]{3.1} is an example of a Research
Questions; Rec. \hyperref[rec:what]{3.1} is an example of a
recommendation.

\phantomsection\label{trq:what}
\begin{fbx}{TRQ}{Thesis Research Question 3.1: }{What is a TRQ?}
\phantomsection\label{trq:what}
This is a Thesis Research Question, that you might want to use in the
introduction of your thesis.

\end{fbx}

\phantomsection\label{rq:what}
\begin{fbx}{RQ}{Research Question 3.1: }{What is a RQ?}
\phantomsection\label{rq:what}
This is a Research Question, that you might want to use in any chapter
of your thesis.

\end{fbx}

\phantomsection\label{rec:what}
\begin{fbx}{Rec}{Recommendation 3.1: }{What is a Rec?}
\phantomsection\label{rec:what}
This is a Recommendation that you might want to use in the conclusion of
your thesis.

\end{fbx}




\end{document}
