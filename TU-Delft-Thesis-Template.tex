% Options for packages loaded elsewhere
\PassOptionsToPackage{unicode}{hyperref}
\PassOptionsToPackage{hyphens}{url}
\PassOptionsToPackage{dvipsnames,svgnames,x11names}{xcolor}
%
\documentclass{tudelft}

\usepackage{amsmath,amssymb}
\usepackage{iftex}
\ifPDFTeX
  \usepackage[T1]{fontenc}
  \usepackage[utf8]{inputenc}
  \usepackage{textcomp} % provide euro and other symbols
\else % if luatex or xetex
  \usepackage{unicode-math}
  \defaultfontfeatures{Scale=MatchLowercase}
  \defaultfontfeatures[\rmfamily]{Ligatures=TeX,Scale=1}
\fi
\usepackage{lmodern}
\ifPDFTeX\else  
    % xetex/luatex font selection
\fi
% Use upquote if available, for straight quotes in verbatim environments
\IfFileExists{upquote.sty}{\usepackage{upquote}}{}
\IfFileExists{microtype.sty}{% use microtype if available
  \usepackage[]{microtype}
  \UseMicrotypeSet[protrusion]{basicmath} % disable protrusion for tt fonts
}{}
\makeatletter
\@ifundefined{KOMAClassName}{% if non-KOMA class
  \IfFileExists{parskip.sty}{%
    \usepackage{parskip}
  }{% else
    \setlength{\parindent}{0pt}
    \setlength{\parskip}{6pt plus 2pt minus 1pt}}
}{% if KOMA class
  \KOMAoptions{parskip=half}}
\makeatother
\usepackage{xcolor}
\setlength{\emergencystretch}{3em} % prevent overfull lines
\setcounter{secnumdepth}{5}
% Make \paragraph and \subparagraph free-standing
\makeatletter
\ifx\paragraph\undefined\else
  \let\oldparagraph\paragraph
  \renewcommand{\paragraph}{
    \@ifstar
      \xxxParagraphStar
      \xxxParagraphNoStar
  }
  \newcommand{\xxxParagraphStar}[1]{\oldparagraph*{#1}\mbox{}}
  \newcommand{\xxxParagraphNoStar}[1]{\oldparagraph{#1}\mbox{}}
\fi
\ifx\subparagraph\undefined\else
  \let\oldsubparagraph\subparagraph
  \renewcommand{\subparagraph}{
    \@ifstar
      \xxxSubParagraphStar
      \xxxSubParagraphNoStar
  }
  \newcommand{\xxxSubParagraphStar}[1]{\oldsubparagraph*{#1}\mbox{}}
  \newcommand{\xxxSubParagraphNoStar}[1]{\oldsubparagraph{#1}\mbox{}}
\fi
\makeatother


\providecommand{\tightlist}{%
  \setlength{\itemsep}{0pt}\setlength{\parskip}{0pt}}\usepackage{longtable,booktabs,array}
\usepackage{calc} % for calculating minipage widths
% Correct order of tables after \paragraph or \subparagraph
\usepackage{etoolbox}
\makeatletter
\patchcmd\longtable{\par}{\if@noskipsec\mbox{}\fi\par}{}{}
\makeatother
% Allow footnotes in longtable head/foot
\IfFileExists{footnotehyper.sty}{\usepackage{footnotehyper}}{\usepackage{footnote}}
\makesavenoteenv{longtable}
\usepackage{graphicx}
\makeatletter
\newsavebox\pandoc@box
\newcommand*\pandocbounded[1]{% scales image to fit in text height/width
  \sbox\pandoc@box{#1}%
  \Gscale@div\@tempa{\textheight}{\dimexpr\ht\pandoc@box+\dp\pandoc@box\relax}%
  \Gscale@div\@tempb{\linewidth}{\wd\pandoc@box}%
  \ifdim\@tempb\p@<\@tempa\p@\let\@tempa\@tempb\fi% select the smaller of both
  \ifdim\@tempa\p@<\p@\scalebox{\@tempa}{\usebox\pandoc@box}%
  \else\usebox{\pandoc@box}%
  \fi%
}
% Set default figure placement to htbp
\def\fps@figure{htbp}
\makeatother
% definitions for citeproc citations
\NewDocumentCommand\citeproctext{}{}
\NewDocumentCommand\citeproc{mm}{%
  \begingroup\def\citeproctext{#2}\cite{#1}\endgroup}
\makeatletter
 % allow citations to break across lines
 \let\@cite@ofmt\@firstofone
 % avoid brackets around text for \cite:
 \def\@biblabel#1{}
 \def\@cite#1#2{{#1\if@tempswa , #2\fi}}
\makeatother
\newlength{\cslhangindent}
\setlength{\cslhangindent}{1.5em}
\newlength{\csllabelwidth}
\setlength{\csllabelwidth}{3em}
\newenvironment{CSLReferences}[2] % #1 hanging-indent, #2 entry-spacing
 {\begin{list}{}{%
  \setlength{\itemindent}{0pt}
  \setlength{\leftmargin}{0pt}
  \setlength{\parsep}{0pt}
  % turn on hanging indent if param 1 is 1
  \ifodd #1
   \setlength{\leftmargin}{\cslhangindent}
   \setlength{\itemindent}{-1\cslhangindent}
  \fi
  % set entry spacing
  \setlength{\itemsep}{#2\baselineskip}}}
 {\end{list}}
\usepackage{calc}
\newcommand{\CSLBlock}[1]{\hfill\break\parbox[t]{\linewidth}{\strut\ignorespaces#1\strut}}
\newcommand{\CSLLeftMargin}[1]{\parbox[t]{\csllabelwidth}{\strut#1\strut}}
\newcommand{\CSLRightInline}[1]{\parbox[t]{\linewidth - \csllabelwidth}{\strut#1\strut}}
\newcommand{\CSLIndent}[1]{\hspace{\cslhangindent}#1}

\makeatletter
\@ifpackageloaded{bookmark}{}{\usepackage{bookmark}}
\makeatother
\makeatletter
\@ifpackageloaded{caption}{}{\usepackage{caption}}
\AtBeginDocument{%
\ifdefined\contentsname
  \renewcommand*\contentsname{Table of contents}
\else
  \newcommand\contentsname{Table of contents}
\fi
\ifdefined\listfigurename
  \renewcommand*\listfigurename{List of Figures}
\else
  \newcommand\listfigurename{List of Figures}
\fi
\ifdefined\listtablename
  \renewcommand*\listtablename{List of Tables}
\else
  \newcommand\listtablename{List of Tables}
\fi
\ifdefined\figurename
  \renewcommand*\figurename{Figure}
\else
  \newcommand\figurename{Figure}
\fi
\ifdefined\tablename
  \renewcommand*\tablename{Table}
\else
  \newcommand\tablename{Table}
\fi
}
\@ifpackageloaded{float}{}{\usepackage{float}}
\floatstyle{ruled}
\@ifundefined{c@chapter}{\newfloat{codelisting}{h}{lop}}{\newfloat{codelisting}{h}{lop}[chapter]}
\floatname{codelisting}{Listing}
\newcommand*\listoflistings{\listof{codelisting}{List of Listings}}
\makeatother
\makeatletter
\makeatother
\makeatletter
\@ifpackageloaded{caption}{}{\usepackage{caption}}
\@ifpackageloaded{subcaption}{}{\usepackage{subcaption}}
\makeatother
\makeatletter
\@ifpackageloaded{algorithm}{}{\usepackage{algorithm}}
\makeatother
\makeatletter
\@ifpackageloaded{algpseudocode}{}{\usepackage{algpseudocode}}
\makeatother
\makeatletter
\@ifpackageloaded{caption}{}{\usepackage{caption}}
\makeatother
\makeatletter
\@ifpackageloaded{tcolorbox}{}{\usepackage[many]{tcolorbox}}
\makeatother
%%%% ---foldboxy preamble ----- %%%%%

\definecolor{fbx-default-color1}{HTML}{c7c7d0}
\definecolor{fbx-default-color2}{HTML}{a3a3aa}

\definecolor{fbox-color1}{HTML}{c7c7d0}
\definecolor{fbox-color2}{HTML}{a3a3aa}

% arguments: #1 typelabelnummer: #2 titel: #3
\newenvironment{fbx}[3]{\begin{tcolorbox}[enhanced, breakable,%
attach boxed title to top*={xshift=1.4pt},
boxed title style={boxrule=0.0mm, fuzzy shadow={1pt}{-1pt}{0mm}{0.1mm}{gray}, arc=.3em, rounded corners=east, sharp corners=west}, colframe=#1-color2, colbacktitle=#1-color1, colback = white, coltitle=black,  titlerule=0mm, toprule=0pt, bottomrule=.7pt, leftrule=.3em, rightrule=0pt, outer arc=.3em,  arc=0pt,	 sharp corners = east, left=.5em, bottomtitle=1mm, toptitle=1mm,title=\textbf{#2}\hspace{0.5em}{#3}]}
{\end{tcolorbox}}

% boxed environment with right border
\newenvironment{fbxSimple}[3]{\begin{tcolorbox}[enhanced, breakable,%
attach boxed title to top*={xshift=1.4pt},
boxed title style={boxrule=0.0mm, fuzzy shadow={1pt}{-1pt}{0mm}{0.1mm}{gray}, arc=.3em, rounded corners=east, sharp corners=west}, colframe=#1-color2, colbacktitle=#1-color1, colback = white, coltitle=black,  titlerule=0mm, toprule=0pt, bottomrule=.7pt, leftrule=.3em, rightrule=.7pt, outer arc=.3em,  	left=.5em, right=.5em, bottomtitle=1mm, toptitle=1mm,title=\textbf{#2}\hspace{0.5em}{#3}]}
{\end{tcolorbox}}

%%%% --- end foldboxy preamble ----- %%%%%
%%==== colors from yaml ===%
\definecolor{Rec-color1}{HTML}{fff6e3}
\definecolor{Rec-color2}{HTML}{ffb81c}
\definecolor{TRQ-color1}{HTML}{80d2eb}
\definecolor{TRQ-color2}{HTML}{00a6d6}
\definecolor{RQ-color1}{HTML}{dff4fa}
\definecolor{RQ-color2}{HTML}{00a6d6}
%=============%

\usepackage{bookmark}

\IfFileExists{xurl.sty}{\usepackage{xurl}}{} % add URL line breaks if available
\urlstyle{same} % disable monospaced font for URLs
\hypersetup{
  pdftitle={TU Delft Thesis Template},
  pdfauthor={Patrick Altmeyer},
  colorlinks=true,
  linkcolor={blue},
  filecolor={Maroon},
  citecolor={Blue},
  urlcolor={Blue},
  pdfcreator={LaTeX via pandoc}}


\title{TU Delft Thesis Template}
\author{Patrick Altmeyer}
\date{2025-05-13}

\begin{document}
% !TeX spellcheck = en_GB
\begin{titlepage}

    \begin{center}

        %% Extra whitespace at the top.
        \vspace*{2\bigskipamount}

        %% Print the title.
        {\makeatletter
            \titlestyle\bfseries\LARGE\@title
            \makeatother}

        %% Print the optional subtitle.
        {\makeatletter
            \ifx\@subtitle\undefined\else
                \bigskip
                \titlefont\titleshape\Large\@subtitle
            \fi
            \makeatother}

    \end{center}

    \cleardoublepage
    \thispagestyle{empty}

    \begin{center}

        %% The following lines repeat the previous page exactly.

        \vspace*{2\bigskipamount}

        %% Print the title.
        {\makeatletter
            \titlestyle\bfseries\LARGE\@title
            \makeatother}

        %% Print the optional subtitle.
        {\makeatletter
            \ifx\@subtitle\undefined\else
                \bigskip
                \titlefont\titleshape\Large\@subtitle
            \fi
            \makeatother}

        %% Uncomment the following lines to insert a vertically centered picture into
        %% the title page.
        %\vfill
        %\includegraphics{title}
        \vfill

        %% Apart from the names and dates, the following text is dictated by the
        %% promotieregelement.

        {\Large\titlefont\bfseries Dissertation}

        \bigskip
        \bigskip

        for the purpose of obtaining the degree of doctor

        at Delft University of Technology

        by the authority of the Rector Magnificus, prof.~dr.~ir.~T.H.J.J.~van der Hagen,

        chair of the Board for Doctorates

        to be defended publicly on

            [date= weekday (word) day (number), month (word) year (number)] at [hh:mm (number)] o'clock

        \bigskip
        \bigskip

        by

        \bigskip
        \bigskip

        %% Print the full name of the author.
        \makeatletter
        % {\Large\titlefont\bfseries\@firstnames\ \MakeUppercase{\titleshape\@lastname}}
        \makeatother

        \bigskip
        \bigskip

        [highest academic title, name university, country]

        born in [town/city, country of birth]

        %% Extra whitespace at the bottom.
        \vspace*{2\bigskipamount}

    \end{center}

    \clearpage
    \thispagestyle{empty}

    %% The following line is dictated by the promotieregelement.
    \noindent This dissertation has been approved by the promotors.

    \bigskip
    \noindent Composition of the doctoral committee:
    %% List the committee members, starting with the Rector Magnificus and the
    %% promotor(s) and ending with the reserve members.
    \begin{tabbing}
        \hspace{\tabcolsep}\=\hspace{0.33\textwidth}\=\hspace{0.66\textwidth}                   \\[-3\medskipamount]
        \> Rector Magnificus,          \> chairperson\\
        \> Prof.\ dr.\ A.\ Kleiner,    \> Delft University of Technology, \textit{promotor}      \\
        \> Dr.\ A.A.\ Aaronson,        \> Delft University of Technology, \textit{copromotor}    \\[\medskipamount]
        \>\textit{Independent members:}                                                        \\[\smallskipamount]
        \>Prof.\ dr.\ A.\ Jansen       \> Delft University of Technology                         \\
        % Special case, only for very long names
        \>Prof.\ dr.\ ir.\ A.B.C.D.\ van de Lange-Achternaam                                    \\
        \>                             \> Delft University of Technology                         \\
        \>Prof.\ dr.\ N.\ Nescio       \> Politecnico di Milano, Italy                       \\
        \>Prof.\ dr.\ ir.\ J.\ Doe,    \> Delft University of Technology, reserve member             \\[\medskipamount]
        \>\textit{Other members:}                                                               \\[\smallskipamount]
        \>Prof.\ dr.\ ir.\ J.\ de Wit, \> Delft University of Technology                         \\
        \>Dr.\ ir.\ Q.\ de Zwart,      \> Delft University of Technology\\
    \end{tabbing}

    %% Include the following disclaimer for committee members who have contributed
    %% to this dissertation. Its formulation is again dictated by the
    %% promotieregelement.
    \medskip
    \noindent Prof.\ dr.\ ir.\ J.\ de Wit of Delft University of Technology has contributed greatly to the preparation of this dissertation.

    %% Here you can include the logos of any institute that contributed financially
    %% to this dissertation.
    \vfill
    \begin{center}
        % \includegraphics[height=0.5in]{/www/TUDelft_logo_rgb.png}
    \end{center}
    \vfill

    \noindent
    \begin{tabular}{@{}p{0.2\textwidth}@{}p{0.8\textwidth}@{}}
        \textit{Keywords:}    & \ldots                                                                                        \\[\medskipamount]
        \textit{Printed by:}   & Johannes Gutenberg                                                                            \\[\medskipamount]
        \textit{Cover by:} & Beautiful cover art that captures the entire content of this thesis in a single illustration.
    \end{tabular}

    \vspace{4\bigskipamount}

    \noindent Copyright \textcopyright{} \the\year{} by{
        \makeatletter
        % \@initials~\@lastname
        \makeatother
    }

    %% Uncomment the following lines if this dissertation is part of the Casimir PhD
    %% Series, or a similar research school.
    %\medskip
    %\noindent Casimir PhD Series, Delft-Leiden 2015-01

    \medskip
    \noindent ISBN 000-00-0000-000-0

    \medskip
    \noindent An electronic copy of this dissertation is available at\\
    \url{https://repository.tudelft.nl/}.

\end{titlepage}
\floatname{algorithm}{Algorithm}

\numberwithin{algorithm}{chapter}

\renewcommand*\contentsname{Table of contents}
{
\hypersetup{linkcolor=}
\setcounter{tocdepth}{2}
\tableofcontents
}

\bookmarksetup{startatroot}

\chapter*{Preface}\label{preface}
\addcontentsline{toc}{chapter}{Preface}

\markboth{Preface}{Preface}

This is a Quarto book.

To learn more about Quarto books visit
\url{https://quarto.org/docs/books}.

\bookmarksetup{startatroot}

\chapter{Introduction}\label{introduction}

This is a book created from markdown and executable code.

See Knuth (1984) for additional discussion of literate programming.

\bookmarksetup{startatroot}

\chapter{Summary}\label{summary}

In summary, this book has no content whatsoever.

\bookmarksetup{startatroot}

\chapter*{References}\label{references}
\addcontentsline{toc}{chapter}{References}

\markboth{References}{References}

\phantomsection\label{refs}
\begin{CSLReferences}{1}{0}
\bibitem[\citeproctext]{ref-knuth84}
Knuth, Donald E. 1984. {``Literate Programming.''} \emph{Comput. J.} 27
(2): 97--111. \url{https://doi.org/10.1093/comjnl/27.2.97}.

\end{CSLReferences}




\end{document}
